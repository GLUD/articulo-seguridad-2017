% Andres ________________________________________
%Principales herramientas de seguridad informática
	%Distribución GNU/LINUX de seguridad informática
    	%Herramientas de software libre.

Kali Linux es una distribución de software libre la cual esta en constante evolución con nuevas características que se agregan constantemente, esta distribución esta diseñada específicamente para pruebas de penetración profesional y auditoria de seguridad.

% Leidy opina que acá debería ir una coma y no un punto aparte ^3^ Gracias mi vida! <3 
Cambios principales de esta distribución:

Usuario único, acceso root por diseño: Es diseñado para trabajar en un único escenario root (usuario principal o administrador del sistema) ya que la mayoría de las herramientas utilizadas en pruebas de penetración requieren privilegios escalonados.

Servicios de red deshabilitados de forma predeterminada: Kali Linux contiene Demonios (scripts en segundo plano) que desactivan los servicios de red de forma predeterminada. Estos Demonios nos permiten instalar varios servicios en Kali Linux, a la vez que nos aseguramos de que nuestra distribución permanezca segura de forma predeterminada, sin importar qué paquetes estén instalados. Los servicios adicionales, como Bluetooth, también están en la lista negra de forma predeterminada.

Núcleo Linux personalizado:

Kali Linux utiliza un kernel en sentido ascendente, con parches para inyección inalambrica.

Un conjunto mínimo y confiable de repositorios:

Dado los objetivos y metas de Kali Linux para mantener la integridad del sistema como un todo. Con ese objetivo en mente, el conjunto de fuentes de software ascendente que Kali utiliza se mantiene en un mínimo absoluto. Muchos nuevos usuarios de Kali están tentados a agregar repositorios adicionales a sus sources.list, pero al hacerlo corre un riesgo muy serio de romper la instalación de Kali Linux.

¿Es Kali Linux adecuado para usted?

Los desarrolladores de Kali linux recomiendan que esta distribución debe ser usada por probadores de penetración profesional y especialistas en seguridad, ya que no es una distribución para soluciones generales de GNU/LINUX como el desarrollo de aplicaciones, diseño web, juegos, etc.

El uso indebido de herramientas de seguridad y de pruebas de penetración dentro de una red, en particular sin una autorización específica, puede causar daños irreparables y tener consecuencias significativas, personales y/o legales. "No entender lo que estabas haciendo" no va a funcionar como una excusa.

Sin embargo, si usted es un probador de penetración profesional o está estudiando pruebas de penetración con el objetivo de convertirse en un profesional certificado, no hay mejor juego de herramientas - a cualquier precio - que Kali Linux.

Algunas Herramientas de seguridad y hacking.

Nmap ("Mapeador de redes")

Es una herramienta de código abierto para la exploración de redes y sondeo de seguridad de puertos, se diseño para analizar rápidamente grandes redes, Utiliza paquetes IP "crudos" para poder determinar que equipos se encuentran disponibles en la red, esta herramienta generalmente se usa para auditoria de seguridad, administradores de redes, y es útil para realizar tareas rutinarias, como puede ser el inventariado de la red, la planificación de actualización de servicios y la motorización del tiempo que los equipos o servicios se mantiene activos en la red.

El código fuente lo puedes encontrar en: https://github.com/nmap/nmap

John the Ripper

John de Ripper es un programa libre de código abierto de criptografia, el cual aplica fuerza bruta para descifrar contraseñas, esta herramienta nos permite romper algoritmos de cifrado o hash, como DES, SHA-1 y otros.

El código fuente lo puedes encontrar en: https://github.com/magnumripper/JohnTheRipper

Nikto

Nikto es un software de código abierto (GPL), diseñado para evaluar la seguridad de los servidores web. Este software encuentra varios archivos predeterminados e inseguros, configuraciones y programas en cualquier tipo de servidor web.

El código fuente lo puedes encontrar en: https://github.com/sullo/nikto

Wireshark

Wireshark es un software de código abierto (GPL) el cual lo usan administradores de redes para analizar el trafico de la red, capturar paquetes de diferentes tipos de protocolos que existen.

El código fuente lo puedes encontrar en: https://github.com/wireshark/wireshark

Metasploit

Metasploit es un proyecto de código abierto para seguridad informática el cual proporciona información acerca de vulnerabilidades de seguridad y nos ayuda a realizar tests de penetración "Pentesting" junto con el desarrollo de firmas para sistemas de detección de intrusos.

El código fuente lo puedes encontrar en:  https://github.com/rapid7/metasploit-framework

REFERENCIAS

https://docs.kali.org/
https://nmap.org/man/es/
http://www.openwall.com/john/
https://www.wireshark.org/docs/
